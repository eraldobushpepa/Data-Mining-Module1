\stepcounter{chapter}
\chapter*{Data Understanding and Preparation} 
\phantomsection
\addcontentsline{toc}{chapter}{\protect\numberline{\thechapter}Data Understanding and Preparation}
\markboth{Data Understanding and Preparation}{} 

\vspace{-10mm}
\section{Dataset}
The \textit{Board Games Dataset} contains information regarding 
more than 20k board games rated by an online board game 
community. The information provided range from year of publication, 
game difficulty, number of players, information about games rank in 
specific categories and game characteristics/themes.
The variables are specified in the table below:

\begin{footnotesize}
\begin{longtable}{|p{4cm}|p{10cm}|p{1.5cm}|}
\hline
\textbf{Name} & \textbf{Description} & \textbf{Type} \\
\hline
\endhead

\texttt{BGGId} & Game Id & \texttt{int64} \\
\texttt{Name} & Name of game & \texttt{object} \\
\texttt{Description} & Description of the game & \texttt{object} \\
\texttt{YearPublished} & Year in which the game was published & \texttt{int64} \\
\texttt{GameWeight} & Game complexity from 1 to 5 & \texttt{float64} \\
\texttt{ComWeight} & Community recommended game complexity from 1 to 5 & \texttt{float64} \\
\texttt{MinPlayers} & Minimum number of players & \texttt{int64} \\
\texttt{MaxPlayers} & Maximum number of players & \texttt{int64} \\
\texttt{ComAgeRec} & Community's recommended age minimum & \texttt{float64} \\
\texttt{LanguageEase} & Language requirement & \texttt{float64} \\
\texttt{BestPlayers} & Community voted best player count & \texttt{int64} \\
\texttt{GoodPlayers} & List of community voted good player counts & \texttt{object} \\
\texttt{NumOwned} & Number of users who own this game & \texttt{int64} \\
\texttt{NumWant} & Number of users who want this game & \texttt{int64} \\
\texttt{NumWish} & Number of users who wishlisted this game & \texttt{int64} \\
\texttt{NumWeightVotes} & Number of votes for the weight category received by users & \texttt{int64} \\
\texttt{MfgPlaytime} & Manufacturer Stated Play Time & \texttt{int64} \\
\texttt{ComMinPlaytime} & Community minimum play time & \texttt{int64} \\
\texttt{ComMaxPlaytime} & Community maximum play time & \texttt{int64} \\
\texttt{MfgAgeRec} & Manufacturer Age & \texttt{int64} \\
\texttt{NumUserRatings} & Number of user ratings & \texttt{int64} \\
\texttt{NumComments} & Number of user comments & \texttt{int64} \\
\texttt{NumAlternates} & Number of alternate versions & \texttt{int64} \\
\texttt{NumExpansions} & Number of expansions & \texttt{int64} \\
\texttt{NumImplementations} & Number of implementations & \texttt{int64} \\
\texttt{IsReimplementation} & Is this game presenting a reimplementation? & \texttt{int64} \\
\texttt{Family} & Game family the game belongs to & \texttt{object} \\
\texttt{Kickstarted} & Is this game from a kickstarter (crowdfunding campaign) project? & \texttt{int64} \\
\texttt{ImagePath} & Image http:// path & \texttt{object} \\
\texttt{Rank:strategygames} & Rank in strategy games & \texttt{int64} \\
\texttt{Rank:abstracts} & Rank in abstracts & \texttt{int64} \\
\texttt{Rank:familygames} & Rank in family games & \texttt{int64} \\
\texttt{Rank:thematic} & Rank in thematic & \texttt{int64} \\
\texttt{Rank:cgs} & Rank in card games & \texttt{int64} \\
\texttt{Rank:wargames} & Rank in war games & \texttt{int64} \\
\texttt{Rank:partygames} & Rank in party games & \texttt{int64} \\
\texttt{Rank:childrensgames} & Rank in children's games & \texttt{int64} \\
\texttt{Cat:Thematic} & Binary is in Thematic category & \texttt{int64} \\
\texttt{Cat:Strategy} & Binary is in Strategy category & \texttt{int64} \\
\texttt{Cat:War} & Binary is in War category & \texttt{int64} \\
\texttt{Cat:Family} & Binary is in Family category & \texttt{int64} \\
\texttt{Cat:CGS} & Binary is in Card Games category & \texttt{int64} \\
\texttt{Cat:Abstract} & Binary is in Abstract category & \texttt{int64} \\
\texttt{Cat:Party} & Binary is in Party category & \texttt{int64} \\
\texttt{Cat:Childrens} & Binary is in Childrens category & \texttt{int64} \\
\texttt{Rating} & Game rating (low, medium, high) & \texttt{object} \\
\hline
\end{longtable}
\end{footnotesize}

\noindent This report contains the summary of the analysis performed on the dataset in two stages: Data Understanding and Preparation, and Clustering.
From initial exploration we have:

\noindent Total number of records: 21925\\
\noindent Total number of attributes: 46\\
\noindent In our dataset, we found 0 duplicate rows.

\section{Distribution of variables}
An automatic classification of the variables based on their \texttt{dtype} and cardinality was performed. This analysis is crucial for determining the correct preparation strategy (e.g., scaling, transformation).
\begin{itemize}
    \item \textbf{Categorical (18 columns):} Object types, IDs, and binary flags. 

    \textit{\small BGGId, Name, Description, GoodPlayers, NumComments, IsReimplementation, Family, Kickstarted, ImagePath, Cat:Thematic, Cat:Strategy, Cat:War, Cat:Family, Cat:CGS, Cat:Abstract, Cat:Party, Cat:Childrens, Rating}
    
    \item \textbf{Continuous (4 columns):} All \texttt{float64} types. \\
    \textit{\small GameWeight, ComWeight, ComAgeRec, LanguageEase}

    \item \textbf{Discrete (24 columns):} All high-cardinality \texttt{int64} types (counts, ranks, etc.). \\
    \textit{\small YearPublished, MinPlayers, MaxPlayers, BestPlayers, NumOwned, NumWant, NumWish, NumWeightVotes, MfgPlaytime, ComMinPlaytime, ComMaxPlaytime, MfgAgeRec, NumUserRatings, NumAlternates, NumExpansions, NumImplementations, Rank:strategygames, Rank:abstracts, Rank:familygames, Rank:thematic, Rank:cgs, Rank:wargames, Rank:partygames, Rank:childrensgames}
\end{itemize}


\noindent A skewness and kurtosis analysis (Table \ref{tab:skew_kurtosis}) was performed on all numeric features to test for normality.

\begin{table}[H]
\centering
\caption{Skewness and Kurtosis of Numeric Features. High positive values indicate a heavy right-skew and non-normal distributions.}
\label{tab:skew_kurtosis}
\begin{footnotesize}
\begin{tabular}{|l|r|r|}
\hline
\textbf{Feature} & \textbf{Skewness} & \textbf{Kurtosis} \\
\hline
GameWeight & 0.395861 & 0.053182 \\
ComWeight & 0.302567 & 0.209106 \\
ComAgeRec & 0.143862 & -0.381596 \\
LanguageEase & 1.671916 & 4.431087 \\
YearPublished & -11.324235 & 152.995691 \\
MinPlayers & 1.704234 & 10.722395 \\
MaxPlayers & 42.387696 & 2647.275467 \\
BestPlayers & 3.733134 & 15.745030 \\
NumOwned & 12.517373 & 238.628210 \\
NumWant & 6.956857 & 65.837595 \\
NumWish & 9.350407 & 124.392344 \\
NumWeightVotes & 15.317043 & 366.316525 \\
MfgPlaytime & 74.739212 & 7730.566352 \\
ComMinPlaytime & 116.207073 & 15289.798471 \\
ComMaxPlaytime & 74.739212 & 7730.566352 \\
MfgAgeRec & -0.838558 & 0.947437 \\
NumUserRatings & 12.586978 & 231.561002 \\
NumAlternates & 52.601012 & 3822.922060 \\
NumExpansions & 24.951409 & 1186.177390 \\
NumImplementations & 12.157622 & 342.326796 \\
Rank:strategygames & -2.569618 & 4.615674 \\
Rank:abstracts & -4.090590 & 14.739232 \\
Rank:familygames & -2.571919 & 4.627397 \\
Rank:thematic & -3.871507 & 12.995307 \\
Rank:cgs & -8.329871 & 67.394094 \\
Rank:wargames & -1.857211 & 1.471660 \\
Rank:partygames & -5.594612 & 29.305022 \\
Rank:childrensgames & -4.684208 & 19.947404 \\
\hline
\end{tabular}
\end{footnotesize}
\end{table}

\noindent The results in Table \ref{tab:skew_kurtosis} clearly show that most count-based columns (like \texttt{NumOwned} and \texttt{MaxPlayers}) are extremely right-skewed and not normally distributed. This finding justifies our decision to use a log-transform and a non-parametric scaler (\texttt{RobustScaler}) during preparation.

\section{Outliers detection}
\label{outliers_detection}


\noindent To identify outliers, both Z-Score (Table \ref{tab:zscore_outliers}) and the Interquartile Range (IQR) method (Table \ref{tab:iqr_outliers}) were used. Given the non-normal, skewed distribution of our data (seen in Section 1.2), the IQR method is considered more reliable.

\begin{table}[H]
\centering
\caption{Potential Outliers (Z-Score > 3)}
\label{tab:zscore_outliers}
\begin{footnotesize}
\begin{tabular}{|l|r|}
\hline
\textbf{Feature} & \textbf{Outlier Count (Z-Score)} \\
\hline
YearPublished & 218 \\
GameWeight & 51 \\
ComWeight & 38 \\
MinPlayers & 118 \\
MaxPlayers & 190 \\
ComAgeRec & 28 \\
LanguageEase & 206 \\
BestPlayers & 1048 \\
NumOwned & 307 \\
NumWant & 433 \\
NumWish & 353 \\
NumWeightVotes & 279 \\
MfgPlaytime & 67 \\
ComMinPlaytime & 21 \\
ComMaxPlaytime & 67 \\
MfgAgeRec & 13 \\
NumUserRatings & 304 \\
NumAlternates & 78 \\
NumExpansions & 232 \\
NumImplementations & 350 \\
Rank:strategygames & 561 \\
Rank:abstracts & 1115 \\
Rank:familygames & 571 \\
Rank:thematic & 1224 \\
Rank:cgs & 303 \\
Rank:partygames & 640 \\
Rank:childrensgames & 881 \\
Cat:Thematic & 1224 \\
Cat:CGS & 303 \\
Cat:Abstract & 1115 \\
Cat:Party & 640 \\
Cat:Childrens & 881 \\
\hline
\end{tabular}
\end{footnotesize}
\end{table}


\begin{table}[H]
\centering
\caption{Potential Outliers (IQR Method). This method is more robust for our skewed data.}
\label{tab:iqr_outliers}
\begin{footnotesize}
\begin{tabular}{|l|r|}
\hline
\textbf{Feature} & \textbf{Outlier Count (IQR)} \\
\hline
MinPlayers & 6886 \\
NumImplementations & 4873 \\
Rank:wargames & 3530 \\
Cat:War & 3530 \\
NumAlternates & 3477 \\
Kickstarted & 3362 \\
NumUserRatings & 3110 \\
NumWish & 3030 \\
NumWeightVotes & 2938 \\
NumWant & 2910 \\
NumOwned & 2845 \\
IsReimplementation & 2560 \\
Rank:strategygames & 2319 \\
Cat:Strategy & 2319 \\
Cat:Family & 2316 \\
Rank:familygames & 2316 \\
NumExpansions & 2183 \\
BestPlayers & 1981 \\
ComMinPlaytime & 1711 \\
MfgPlaytime & 1463 \\
ComMaxPlaytime & 1463 \\
MaxPlayers & 1340 \\
MfgAgeRec & 1339 \\
Cat:Thematic & 1224 \\
Rank:thematic & 1224 \\
\textbf{YearPublished} & \textbf{1143} \\
Rank:abstracts & 1115 \\
Cat:Abstract & 1115 \\
Cat:Childrens & 881 \\
Rank:childrensgames & 881 \\
Rank:partygames & 640 \\
Cat:Party & 640 \\
Cat:CGS & 303 \\
Rank:cgs & 303 \\
LanguageEase & 257 \\
GameWeight & 134 \\
ComWeight & 100 \\
ComAgeRec & 41 \\
\hline
\end{tabular}
\end{footnotesize}
\end{table}

\noindent The IQR analysis (Table \ref{tab:iqr_outliers}) confirmed the presence of a large number of outliers. The most critical finding was in \texttt{YearPublished}, which showed 1,143 outliers. This was caused by some ancient games (e.g., a minimum value of -3500) and justified our decision to clip this feature before imputation. The plots in Figure \ref{fig:main_plots_hists} visualize these distributions.


\begin{figure}[H]
    \centering 
    
    \begin{subfigure}{0.48\textwidth}
        \centering
        \includegraphics[width=\textwidth]{analysis_plots/boxplot_YearPublished.png} 
        \caption{Boxplot for YearPublished}
        \label{fig:boxplot_year}
    \end{subfigure}
    \hfill 
    \begin{subfigure}{0.48\textwidth}
        \centering
        \includegraphics[width=\textwidth]{analysis_plots/boxplot_MinPlayers.png} 
        \caption{Boxplot for MinPlayers}
        \label{fig:boxplot_minplayers}
    \end{subfigure}
    
    \vspace{5mm} % Add a little space between the rows
    
    \begin{subfigure}{0.48\textwidth}
        \centering
        \includegraphics[width=\textwidth]{analysis_plots/hist_NumOwned.png} % <-- FIXED PATH
        \caption{Histogram for NumOwned (Highly Skewed)} 
        \label{fig:hist_numowned}
    \end{subfigure}
    \hfill 
    \begin{subfigure}{0.48\textwidth}
        \centering
        \includegraphics[width=\textwidth]{analysis_plots/hist_GameWeight.png} % <-- FIXED PATH
        \caption{Histogram for GameWeight (Near-Normal)} 
        \label{fig:hist_gameweight}
    \end{subfigure}
    
    \caption{Boxplots (top) visualizing outliers and Histograms (bottom) visualizing data skewness.}
    \label{fig:main_plots_hists}
\end{figure}

\section{Handling missing values}
The dataset was checked for missing values, with the results summarized in Table \ref{tab:missing_values} and visualized in Figure \ref{fig:missing-heatmap}.

\begin{table}[H]
\centering
\caption{Columns with Missing Values}
\label{tab:missing_values}
\begin{footnotesize}
\begin{tabular}{|l|r|r|}
\hline
\textbf{Column} & \textbf{Missing Count} & \textbf{Missing \%} \\
\hline
Family & 15262 & 69.61\% \\
LanguageEase & 5891 & 26.87\% \\
ComAgeRec & 5530 & 25.22\% \\
ImagePath & 17 & 0.08\% \\
Description & 1 & 0.00\% \\
\hline
\end{tabular}
\end{footnotesize}
\end{table}

\begin{figure}[H]
    \centering
    \includegraphics[width=0.8\textwidth]{analysis_plots/heatmap_missing_values.png} 
    \caption{Heatmap visualizing missing data. A yellow line indicates a missing value.}
    \label{fig:missing-heatmap}
\end{figure}


\noindent Based on this analysis, the following preparation strategy was applied:
\begin{itemize}
    \item \textbf{{Family}:} This column was dropped, as it was 69.61\% empty and thus contained little usable information.
    \item \textbf{{LanguageEase} \& \texttt{ComAgeRec}:} These columns were kept. Dropping them would discard over 25\% of the data. Instead, the missing values were imputed using their respective \textbf{median} value.
    \item \textbf{Other columns:} The few remaining missing values (e.g., in \texttt{ImagePath}) were also imputed.
\end{itemize}

\section{Dependencies and correlations}
\label{sec:correlations}
Finally, the correlation between all numeric features was calculated to identify redundant data.

\begin{figure}[H]
    \centering
    \includegraphics[width=1.0\textwidth]{analysis_plots/heatmap_correlation_matrix.png} 
    \caption{Full Correlation Matrix Heatmap.}
    \label{fig:corr-heatmap} 
\end{figure}

\begin{table}[H]
\centering
\caption{Highly Correlated Pairs (Threshold > 0.8)}
\label{tab:corr-pairs}
\begin{footnotesize}
\begin{tabular}{|l|l|r|}
\hline
\textbf{Feature 1} & \textbf{Feature 2} & \textbf{Correlation} \\
\hline
\texttt{MfgPlaytime} & \texttt{ComMaxPlaytime} & 1.000000 \\
\texttt{Rank:cgs} & \texttt{Cat:CGS} & 0.999992 \\
\texttt{Rank:partygames} & \texttt{Cat:Party} & 0.999962 \\
\texttt{Rank:childrensgames} & \texttt{Cat:Childrens} & 0.999927 \\
\texttt{Rank:abstracts} & \texttt{Cat:Abstract} & 0.999880 \\
\texttt{Rank:thematic} & \texttt{Cat:Thematic} & 0.999854 \\
\texttt{Rank:familygames} & \texttt{Cat:Family} & 0.999421 \\
\texttt{Rank:strategygames} & \texttt{Cat:Strategy} & 0.999415 \\
\texttt{Rank:wargames} & \texttt{Cat:War} & 0.998480 \\
\texttt{GameWeight} & \texttt{ComWeight} & 0.997268 \\
\texttt{NumOwned} & \texttt{NumUserRatings} & 0.985474 \\
\texttt{NumWant} & \texttt{NumWish} & 0.939758 \\
\texttt{NumWeightVotes} & \texttt{NumUserRatings} & 0.917185 \\
\texttt{NumOwned} & \texttt{NumWeightVotes} & 0.874876 \\
\texttt{MfgPlaytime} & \texttt{ComMinPlaytime} & 0.854679 \\
\texttt{ComMinPlaytime} & \texttt{ComMaxPlaytime} & 0.854679 \\
\texttt{NumWish} & \texttt{NumUserRatings} & 0.814348 \\
\hline
\end{tabular}
\end{footnotesize}
\end{table}

\noindent The correlation matrix (Figure \ref{fig:corr-heatmap}) and table (Table \ref{tab:corr-pairs}) revealed significant redundancy. All \texttt{Rank:*} columns were nearly identical to their \texttt{Cat:*} counterparts. Furthermore, \texttt{ComWeight} was 99.7\% correlated with \texttt{GameWeight}. Based on these findings, all redundant columns (all \texttt{Rank:*} columns, \texttt{ComWeight}, \texttt{NumWant}, etc.) were dropped.

\subsection*{Final Preparation Strategy}
With all analysis complete, a final preparation script (\texttt{task\_2\_analysis.py}) was created to perform the following steps:
\begin{enumerate}
    \item \textbf{Clip Outliers:} The extreme negative values in \texttt{YearPublished} were clipped.
    \item \textbf{Drop Columns:} All redundant, high-missing, and text-based columns were dropped.
    \item \textbf{Impute Data:} Missing values for \texttt{YearPublished}, \texttt{ComAgeRec}, and \texttt{LanguageEase} were filled using their medians.
    \item \textbf{Log-Transform:} An automatic skew-detection (threshold > 1.0) was run. All 26 identified skewed columns were transformed using \texttt{np.log1p} to normalize their distributions.
    \item \textbf{Scale Data:} Finally, the data was scaled using \texttt{RobustScaler}. This scaler was chosen over \texttt{StandardScaler} because our analysis in Section 1.2 proved the data is not normally distributed and \texttt{RobustScaler} is not sensitive to the outliers identified in Section 1.3.
\end{enumerate}
This process resulted in the final \texttt{dm1\_prepared\_dataset.csv} file used for clustering.